---
title: “The unspotted *dioptra* of prophecy”. A mirror metaphor in Byzantine literature
    *From*: L. Diamantopoulou, M. Gerolemou (eds.), Mirrors and Mirroring. From Antiquity to the Early Modern Period. London et al. 2020.
subtitle: 
author: Eirini Afentoulidou
date: \today
mainfont: times
bibliography: biblioMirrors.bib
csl: cms-fullnote.csl
---

# “The unspotted *dioptra* of prophecy”. A mirror metaphor in Byzantine literature

Eirini Afentoulidou

After having been defeated by the famous frontiersman Digenis Akrites in the homonymous Medieval Greek poem, the Amazon Maximoú proposes the following deal: since Digenis is the first one to defeat her, she shall be his wife and helper. Digenis responds that he already has a beautiful and noble wife. However, as Maximoú takes off her coat because of the heat and remains with a thin shift, Digenis has intercourse with her, having been overcome by her beauty:

>Maximou’s shift was gossamer-thin,

>and it revealed her limbs as in a mirror

>and her breasts rising just a little above her chest.[^1]

There is plenty to discuss in this episode in terms of gender relations. However, I wish to draw attention to one detail: the use of the mirror-simile. Indeed, the tertium comparationis between Maximoú’s gown and a mirror is that they both allow the viewer to see through, revealing something which is normally concealed. This function of the mirror contradicts ancient and modern optics. Yet, the metaphor of mirror as means to revelation is found throughout the Eastern and Western Middle Ages and beyond.[^3]

Mentions of actual mirrors were infrequent in Byzantine literature and usually had connotations of vanity and deception, mostly associated with women. Mirror metaphors, on the other hand, were quite common.[^2] The terms used for mirrors were κάτοπτρον, ἔσοπτρον, and διόπτρα, the etymology of which is examined elsewere in this volume (Bonati and Reggiani). The first two terms were interchangeable and were used both literally and metaphorically. The term διόπτρα on the other hand was much less common; it was used almost exclusively in metaphors in connection with visionary revelation. In the present paper I will examine the mirror metaphor in the spectrum between notions of participation of the effigy in the original, representability of the original through the medium, and prophetic revelation, closing with the only instance of the mirror as title metaphor known from the Byzantine period.

## I.“The Unspotted Mirror of God's Majesty”: The Original and its Likeness

In praising the Wisdom of God the author of the Book of Wisdom, traditionally identified with King Solomon, writes: “she (sc. the personified Wisdom of God) is the brightness of eternal light, and the unspotted mirror of God's majesty, and the image of his goodness”.[^4] The Wisdom is God’s emanation, and this relation is expressed as a relation between the mirror and the original. The Book of Wisdom enjoyed the status of a canonical Old Testament book throughout the Christian Middle Ages. The mirror metaphor, with or without explicit reference to this passage, was widely used by Byzantine theologians, whose thought was permeated by the platonic (or rather neoplatonic) concept of the supreme original and its manifold earthly manifestations.[^5] In this section I will discuss selected texts, in which the mirror metaphor is employed to express the relation between two entities as a relation between original and reflection. These texts are dogmatic treatises pertaining to issues of Christology, i.e. the question of Christ as God and/or human and as one of the triune Divinity, of the validity of icon veneration, and of hesychasm, or they are moral exhortations and hagiographic works presenting a perfected human as mirror of God, or they elaborate on the theme of literature as mirror of a person, be it the author, be it the hero/heroine.

### Christology

In Christian readings the personification of the Book of Wisdom goes a step further, so that the Wisdom is identified with the person of the Son of God Jesus Christ. Thus the author of the Letter to the Hebrews, in Christian tradition identified with the Apostle Paul, refers to Christ alluding to this passage but leaving the word ἔσοπτρον out: “the brightness of his glory, and the express image of his person”.[^6] Even if the “unspotted mirror” is left out in the Letter to the Hebrews, the mirror analogy in allusion to the Book of Wisdom was occasionally used by Christian authors to define the relation between God the Father and God the Son. The 7th century author Anastasius of Sinai, for example, claims: “In the Son, as in some divine mirror, we see the glory of the Father. So it is written: ‘He who has seen me, has seen the Father (John 14, 9)’”.[^7]

Few Byzantine authors, however, used the mirror analogy in trinitarian context, i.e. to define the relation between the divine persons of the Holy Trinity. The reason must be sought in its implications, which are incompatible with the consubstantiality dogma of the Nicean Creed (325 CE). According to this dogma, the three persons of the Holy Trinity are of the same substance. A mirror, on the other hand, is ontologically different from the original, even if it temporarily bears its reflection. Therefore, the mirror was rather used by theologians for relations, in which the hierarchy was uncontested. A not uncommon occasion was Christ’s incarnation. For example we read in the Physiologus, a symbolic bestiary circulating in various versions since Late Antiquity (the following passage is a Byzantine interpolation): “But God’s Wisdom, i.e. our Lord Jesus Christ, descended from heaven and shone in the world through his flesh, which he took from the holy maiden and God-bearer Mary, as in a mirror.”[^8] A mirror of Christ’s divine power are also his works: “From what the Son did before and after incarnation, his infinite power appears as in a mirror”.[^9]

These examples are characteristic in as much as they demonstrate the emphasis Byzantine theologians put on the problem of the perceptibility of God. Indeed, the aim of the mirror analogy is to show how what we see (Christ, Christ’s body, Christ’s deeds) relates to what we cannot see (God the Father, Christ’s divinity). The question of the perceptibility of the Divine culminated in the iconoclastic and hesychastic controversies, to which the mirror analogy almost lends itself.[^10]



[^1: ]Much of the material for this article was gathered in the framework of the now finished project “Dioptra. Edition der griechischen Version”, financed by the Austrian Science Fund (FWF) (Einzelprojekte P21811). I wish once again to thank Wolfram Hörandner for the long inspiring discussions on the subject.

Καὶ ὁ χιτὼν τῆς Μαξιμοῦς ὑπῆρχεν ἀραχνώδης·/ πάντα καθάπερ ἔσοπτρον ἐνέφαινε τὰ μέλη/ καὶ τοὺς μαστοὺς προκύπτοντας μικρὸν ἄρτι τῶν στέρνων. Edited and trans. by Jeffreys 1998, Grottaferrata Version 6, 783. The Grottaferrate version is dated into the 12th century.
[^2]: See for example Anderson 2007.
[^3]: See Papaioannou 2010, 81–101.
[^4]: Ἀπαύγασμα γάρ ἐστιν φωτὸς ἀιδίου καὶ ἔσοπτρον ἀκηλίδωτον τῆς τοῦ θεοῦ ἐνεργείας καὶ εἰκὼν τῆς ἀγαθότητος αὐτοῦ. Sapientia 7, 26, Douay–Rheims translation.
[^5]: See Benakis 1982, 75-86.
[^6]: Ὃς ὢν ἀπαύγασμα τῆς δόξης καὶ χαρακτὴρ τῆς ὑποστάσεως αὐτοῦ. Hebrews 1, 3. Trans.: King James Version.
[^7]: Ἐν τῷ Υἱῷ ὡς ἐν ἐσόπτρῳ θείῳ τινὶ ὁρῶμεν τὴν δόξαν τοῦ Πατρὸς κατὰ τό· Ὁ ἑωρακὼς ἐμὲ ἑώρακε τὸν Πατέρα. Ed. and trans. by Kuehn – Baggarly 2007, II 140-142.
[^8]: Ἀλλ’ ἡ σοφία τοῦ Θεοῦ, ἤγουν ὁ Κύριος ἡμῶν Ἰησοῦς Χριστός, κατελθὼν ἐκ τῶν οὐρανῶν καὶ ὡς ἐν ἐσόπτρῳ διὰ σαρκὸς λάμψας ἐν κόσμῳ, ἣν ἔλαβεν ἐκ τῆς ἁγίας θεόπαιδος καὶ θεοτόκου Μαρίας. Ed. by Sbordone 1936, Appendix, p. 316, 30-33. The translation is mine.
[^9]: Ἐξ ὧν πεποίηκεν ὁ υἱὸς καὶ πρὸ σαρκώσεως καὶ μετὰ σάρκωσιν, φαίνεται κατὰ νοῦν ὡς «ἐν ἐσόπτρῳ» ἡ ἄπειρος αὐτοῦ ἰσχύς. Reuss 1966, Fragment 485 (attributed to Ammonius). The translation is mine.
[^10]: Avenarius 1998, English translation 2005.

## Bibliography